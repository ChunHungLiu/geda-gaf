% gEDA - GPL Electronic Design Automation
% fileformats.tex - gEDA/gaf File Formats 
% Copyright (C) 2002 Ales V. Hvezda
%
% This program is free software; you can redistribute it and/or modify
% it under the terms of the GNU General Public License as published by
% the Free Software Foundation; either version 2 of the License, or
% (at your option) any later version.
%
% This program is distributed in the hope that it will be useful,
% but WITHOUT ANY WARRANTY; without even the implied warranty of
% MERCHANTABILITY or FITNESS FOR A PARTICULAR PURPOSE.  See the
% GNU General Public License for more details.
%
% You should have received a copy of the GNU General Public License
% along with this program; if not, write to the Free Software
% Foundation, Inc., 675 Mass Ave, Cambridge, MA 02139, USA.

\documentclass{article}
\usepackage{epsfig}

\setlength{\parindent}{0pt}
\setlength{\parskip}{1ex plus 0.5ex minus 0.2ex}

\title{gEDA/gaf File Format Document}
\author{Ales V. Hvezda, ahvezda@geda.seul.org\\
        \\
        This document is released under GFDL\\
        (http://www.gnu.org/copyleft/fdl.html)}
\date{November 30th, 2002}

\begin{document}

\maketitle
\newpage

\tableofcontents
\newpage


\section{Overview}

This file is the official documentation for the file formats in gEDA/gaf
(Gschem And Friends).  The primary file format used in gEDA/gaf is the
schematic/symbol format.  Files which end with .sch or .sym are schematics
or symbol files. This file format is current as of gEDA version 20021103.
Until there is another file type in gEDA/gaf, then this document will
only cover the symbol/schematic file format.


\section{Coordinate Space}
\begin{itemize}
 \item All coordinates are in mils (1/1000 or an inch).  This is an arbitrary decision. Remember in there is no concept of physical lengths/dimensions in schematics
and symbols (for schematic capture only).  
 \item Origin is in lower left hand corner.
 \item The size of the coordinate space is unlimited, but it is recommended that all objects stay within (120.0, 90.0) (x, y inches).
 \item It is generally advisable to have positive x and y coordinates, however, negative coordinates work too, but not recommended.
\end{itemize}

The following figure shows how the coordinate space is setup:

\begin{center}
\epsfbox{coords.eps}
\end{center}

X axis increases going to the right.  Y axis increase going up.
Coordinate system is landscape and corresponds to a sheet of paper turned
on its side.


\section{Filenames}

Symbols end in .sym and schematics end in .sch.  The only filename convention 
that is used in gEDA/gaf is that if there are multiple instances of a symbol
with the same name (like a 7400), then a -1, -2, -3, ... -N suffix is added
to the end of the filename.  Example: 7400-1.sym, 7400-2.sym, 7400-3.sym...

\section{Object types}

A schematic/symbol file for gEDA/gaf consists of:

\begin{itemize}
 \item A version (v) as the first item in the file.  This is required.
 \item Any number of objects and the correct data.  Objects are specified 
       by an "object type"
 \item Most objects are a single line, however text objects are two lines long.
 \item No blank lines at the end of the file (these are ignored by the tools)
 \item For all enumerated types in the gEDA/gaf file formats, the field takes
       on the numeric value.
\end{itemize}

The "object type" id is a single letter and this id {\bf must} start in the 
first column.  The object type id is case sensitive.

The following sections describe the specifics of each recognized object 
type.  Each section has the name of the object, which file type (sch/sym)
the object can appear in, the format of the data, a description of
each individual field, details and caveats of the fields, and finally
an example.

\subsection{version}

Valid in: Schematic and Symbol files

{\bf type version}

\begin{table}[h]
\begin{tabular}{|l|l|} \hline
Field & Description \\ \hline \hline
{\bf type} & v \\ \hline
{\bf version} & version of gEDA/gaf file \\ \hline
\end{tabular}
\end{table}

\begin{itemize}
\item The type is a lower case "v" (as in Victor).
\item This object must be in every file used or created by the gEDA/gaf tools.
\item The format of the version is YYYYMMDD.  
\item The version number is not an arbitrary timestamp.  Do not make up a 
      version number and expect the tools to behave properly.  
\item The version number is used to keep track of file format variations.
\end{itemize}

Valid versions include: \newline
19990601, 19990610, 19990705, 19990829, 19990919, 19991011, 20000220, 20000704,
20001006, 20001217, 20010304, 20010708, 20010722, 20020209, 20020414, 20020527,
20020825, 20021103

Keep in mind that each of the listed versions might have had file format
variations.  This document only covers the last version's file format.

Example: \newline
{\tt v 20021103}


\subsection{line}

Valid in: Schematic and Symbol files

{\bf type x1 y1 x2 y2 color width capstyle dashstyle dashlength dashspace}

\begin{table}[h]
\begin{tabular}{|l|l|} \hline
Field & Description \\ \hline \hline
{\bf type} & L \\ \hline
{\bf x1} & First X coordinate (mils) \\ \hline
{\bf y1} & First Y coordinate (mils) \\ \hline
{\bf x2} & Second X coordinate (mils) \\ \hline
{\bf y2} & Second Y coordinate (mils) \\ \hline
{\bf color} & Color index (See color section) \\ \hline
{\bf width} & Width of line (mils) \\ \hline
{\bf capstyle} & Line cap style: 0, 1, or 2. \\ \hline
{\bf dashstyle} & Type of dash style: 0, 1, 2, 3, or 4 \\ \hline
{\bf dashlength} & Length of dash (mils) \\ \hline
{\bf dashspace} & Space inbetween dashes (mils) \\ \hline
\end{tabular}
\end{table}

\begin{itemize}
\item The capstyle is an enumerated type: 
\begin{itemize}
	\item END\_NONE = 0
	\item END\_SQUARE = 1
	\item END\_ROUND = 2
\end{itemize}
\item The dashstyle is an enumerated type: 
\begin{itemize}
	\item TYPE\_SOLID = 0 
	\item TYPE\_DOTTED = 1
	\item TYPE\_DASHED = 2
	\item TYPE\_CENTER = 3
        \item TYPE\_PHANTOM = 4
\end{itemize}
\item The dashlength parameter is not used for TYPE\_SOLID and TYPE\_DOTTED.  
      This parameter should take on a value of -1 in these cases.
\item The dashspace paramater is not used for TYPE\_SOLID.
      This parameter should take on a value of -1 in these case.
\end{itemize}

Example:\newline
{\tt L 23000 69000 28000 69000 3 40 0 1 -1 75}

A line segment from (23000, 69000) to (28000, 69000) with color index 3, 
40 mils thick, no cap, dotted line style, and with a spacing of 75 mils 
in between each dot.


\subsection{box}

Valid in: Schematic and Symbol files

{\bf type x y width height color width capstyle dashtype dashlength dashspace filltype fillwidth angle1 pitch1 angle2 pitch2 }

\begin{table}[h]
\begin{tabular}{|l|l|} \hline
Field & Description \\ \hline \hline
{\bf type} & B \\ \hline
{\bf x} & Lower left hand X coordinate (mils) \\ \hline 
{\bf y} & Lower left hand Y coordinate  (mils)\\ \hline
{\bf width} & Width of the box (x direction) (mils) \\ \hline
{\bf height} & Height of the box (y direction) (mils) \\ \hline
{\bf color} & Color index (See color section) \\ \hline
{\bf width} & Width of lines (mils) \\ \hline
{\bf capstyle} & Line cap style: 0, 1, or 2. \\ \hline
{\bf dashstyle} & Type of dash style: 0, 1, 2, 3, or 4 \\ \hline
{\bf dashlength} & Length of dash (mils) \\ \hline
{\bf dashspace} & Space inbetween dashes (mils) \\ \hline
{\bf filltype} & Type of fill \\ \hline
{\bf fillwidth} & Width of the fill lines (mils) \\ \hline
{\bf angle1} & First angle of fill (degrees) \\ \hline
{\bf pitch1} & First pitch/spacing of fill (mils) \\ \hline
{\bf angle2} & Second angle of fill (degrees) \\ \hline
{\bf pitch2} & Second pitch/spacing of fill (mils) \\ \hline
\end{tabular}
\end{table}

\begin{itemize}
\item The capstyle is an enumerated type: 
\begin{itemize}
	\item END\_NONE = 0
	\item END\_SQUARE = 1
	\item END\_ROUND = 2
\end{itemize}
\item The dashstyle is an enumerated type: 
\begin{itemize}
	\item TYPE\_SOLID = 0 
	\item TYPE\_DOTTED = 1
	\item TYPE\_DASHED = 2
	\item TYPE\_CENTER = 3
        \item TYPE\_PHANTOM = 4
\end{itemize}
\item The dashlength parameter is not used for TYPE\_SOLID and TYPE\_DOTTED.  
      This parameter should take on a value of -1 in these cases.
\item The dashspace paramater is not used for TYPE\_SOLID.
      This parameter should take on a value of -1 in these case.
\item The filltype parameter is an enumerated type: 
\begin{itemize}
	\item FILLING\_HOLLOW = 0
	\item FILLING\_FILL = 1
	\item FILLING\_MESH = 2 
	\item FILLING\_HATCH = 3
        \item FILLING\_VOID = 4  {\bf unused}
\end{itemize}
\item If the filltype is 0 (FILLING\_HOLLOW), then all the fill parameters 
      should take on a value of -1.
\item The fill type FILLING\_FILL is a solid color fill.
\item The two pairs of pitch and spacing control the fill or hatch if the
      fill type is FILLING\_MESH. 
\item Only the first pair of pitch and spacing are used if the fill type is
      FILLING\_HATCH.
\end{itemize}

Example:\newline
{\tt B 33000 67300 2000 2000 3 60 0 2 75 50 0 -1 -1 -1 -1 -1}

A box with the lower left hand corner at (33000, 67300) and a width and height
of (2000, 2000), color index 3, line width of 60 mils, no cap, dashed line 
type, dash length of 75 mils, dash spacing of 50 mils, no fill, rest parameters 
unset.

\subsection{circle}

Valid in: Schematic and Symbol files

{\bf type x y radius color width capstyle dashtype dashlength dashspace filltype fillwidth angle1 pitch1 angle2 pitch2 }

\begin{table}[h]
\begin{tabular}{|l|l|} \hline
Field & Description \\ \hline \hline
{\bf type} & V \\ \hline
{\bf x} & Center X coordinate (mils) \\ \hline 
{\bf y} & Center Y coordinate (mils)\\ \hline
{\bf radius} & Radius of the circle (mils) \\ \hline
{\bf color} & Color index (See color section) \\ \hline
{\bf width} & Width of circle line (mils) \\ \hline
{\bf capstyle} & Should be 0 {\bf unused} \\ \hline
{\bf dashstyle} & Type of dash style: 0, 1, 2, 3, or 4 \\ \hline
{\bf dashlength} & Length of dash (mils) \\ \hline
{\bf dashspace} & Space inbetween dashes (mils) \\ \hline
{\bf filltype} & Type of fill \\ \hline
{\bf fillwidth} & Width of the fill lines (mils) \\ \hline
{\bf angle1} & First angle of fill (degrees) \\ \hline
{\bf pitch1} & First pitch/spacing of fill (mils) \\ \hline
{\bf angle2} & Second angle of fill (degrees) \\ \hline
{\bf pitch2} & Second pitch/spacing of fill (mils) \\ \hline
\end{tabular}
\end{table}

\begin{itemize}
\item The dashstyle is an enumerated type: 
\begin{itemize}
	\item TYPE\_SOLID = 0 
	\item TYPE\_DOTTED = 1
	\item TYPE\_DASHED = 2
	\item TYPE\_CENTER = 3
        \item TYPE\_PHANTOM = 4
\end{itemize}
\item The dashlength parameter is not used for TYPE\_SOLID and TYPE\_DOTTED.  
      This parameter should take on a value of -1 in these cases.
\item The dashspace paramater is not used for TYPE\_SOLID.
      This parameter should take on a value of -1 in these case.
\item The filltype parameter is an enumerated type: 
\begin{itemize}
	\item FILLING\_HOLLOW = 0
	\item FILLING\_FILL = 1
	\item FILLING\_MESH = 2 
	\item FILLING\_HATCH = 3
        \item FILLING\_VOID = 4  {\bf unused}
\end{itemize}
\item If the filltype is 0 (FILLING\_HOLLOW), then all the fill parameters 
      should take on a value of -1.
\item The fill type FILLING\_FILL is a solid color fill.
\item The two pairs of pitch and spacing control the fill or hatch if the
      fill type is FILLING\_MESH. 
\item Only the first pair of pitch and spacing are used if the fill type is
      FILLING\_HATCH.
\end{itemize}

Example:\newline
{\tt V 38000 67000 900 3 0 0 2 75 50 2 10 20 30 90 50}

A circle with the center at (38000, 67000) and a radius of 900 mils, color 
index 3, line width of 0 mils (smallest size), no cap, dashed line 
type, dash length of 75 mils, dash spacing of 50 mils, mesh fill, 10 mils
thick mesh lines, first mesh line: 20 degrees, with a spacing of 30 mils, 
second mesh line: 90 degrees, with a spacing of 50 mils.


\subsection{arc}

Valid in: Schematic and Symbol files

{\bf type x y radius startangle sweepangle color width capstyle dashtype dashlength dashspace }

\begin{table}[h]
\begin{tabular}{|l|l|} \hline
Field & Description \\ \hline \hline
{\bf type} & A \\ \hline
{\bf x} & Center X coordinate (mils) \\ \hline 
{\bf y} & Center Y coordinate (mils)\\ \hline
{\bf radius} & Radius of the arc (mils) \\ \hline
{\bf startangle} & Starting angle of the arc (degrees) \\ \hline
{\bf sweepangle} & Amount the arc sweeps (degrees) \\ \hline
{\bf color} & Color index (See color section) \\ \hline
{\bf width} & Width of circle line (mils) \\ \hline
{\bf capstyle} & Cap style: 0, 1, or 2. \\ \hline
{\bf dashstyle} & Type of dash style: 0, 1, 2, 3, or 4 \\ \hline
{\bf dashlength} & Length of dash (mils) \\ \hline
{\bf dashspace} & Space inbetween dashes (mils) \\ \hline
\end{tabular}
\end{table}

\begin{itemize}
\item The capstyle is an enumerated type: 
\begin{itemize}
	\item END\_NONE = 0
	\item END\_SQUARE = 1
	\item END\_ROUND = 2
\end{itemize}
\item The dashstyle is an enumerated type: 
\begin{itemize}
	\item TYPE\_SOLID = 0 
	\item TYPE\_DOTTED = 1
	\item TYPE\_DASHED = 2
	\item TYPE\_CENTER = 3
        \item TYPE\_PHANTOM = 4
\end{itemize}
\item The dashlength parameter is not used for TYPE\_SOLID and TYPE\_DOTTED.  
      This parameter should take on a value of -1 in these cases.
\item The dashspace paramater is not used for TYPE\_SOLID.
      This parameter should take on a value of -1 in these case.
\end{itemize}

Example:\newline
{\tt A 30600 75000 2000 0 45 3 0 0 3 75 50}

An arc with the center at (30600, 75000) and a radius of 2000 mils, a 
starting angle of 0, sweeping 45 degrees, color index 3, line width of 0 mils 
(smallest size), no cap, center line type, dash length of 75 mils, dash 
spacing of 50 mils.


\subsection{text}

Valid in: Schematic and Symbol files

{\bf type x y color size visibility show\_name\_value angle alignment}\newline
{\bf string}

\begin{table}[h]
\begin{tabular}{|l|l|} \hline
Field & Description \\ \hline \hline
{\bf type} & T \\ \hline
{\bf x} & First X coordinate (mils) \\ \hline
{\bf y} & First Y coordinate (mils) \\ \hline
{\bf color} & Color index (See color section) \\ \hline
{\bf size} & Size of text (points, 1/72 of an inch) \\ \hline
{\bf visibility} & Visibility of text either 0 or 1 \\ \hline
{\bf show\_name\_value} & Attribute visibility control, either 0, 1, or 2 \\ \hline
{\bf angle} & Angle of the text, either 0, 90, 180, or 270 \\ \hline
{\bf alignment} & Alignment/origin of the text (see below) \\ \hline
{\bf string} & The text string, on a seperate line \\ \hline
\end{tabular}
\end{table}

\begin{itemize}
\item This object is a multi line object.  The first line contains all the 
      text parameters and the second line is the text string.
\item The minimum size is 2 points.
\item There is no maximum size.
\item The coordinate pair is the origin of the text item.
\item The visibility field is an enumerated type:
\begin{itemize}
	\item INVISIBLE = 0 
	\item VISIBLE = 1
\end{itemize}
\item The show\_name\_value is an enumerated type:
\begin{itemize}
	\item SHOW\_NAME\_VALUE = 0  (show both name and value of an attribute)
	\item SHOW\_VALUE = 1 (show only the value of an attribute)
	\item SHOW\_NAME = 2  (show only the name of an attribute)
\end{itemize}
\item The show\_name\_value field is only valid if the string is an attribute
      (string has to be in the form: name=value to be considered an attribute).
\item The angle of the text can only take on one of the following values: 
      0, 90, 180, 270.  A value of 270 will always generate upright text.
\item The alignment/origin field controls the relative location of the 
      origin.
\item The alignment field can take a value from 0 to 8.

The following diagram shows what the values for the alignment field mean:

\begin{center}
\epsfbox{alignment.eps}
\end{center}
\end{itemize}

Example:\newline 
{\tt T 16900 35800 3 10 1 0 0 0}\newline
{\tt Text string!}

A text object with the origin at (16900, 35800), color index 3, 10 points in
size, visible, attribute flags not valid (not an attribute), origin at lower
left, string: Text string!

\newpage
\section{Document Revision History}

\begin{table}[h]
\begin{tabular}{|l|l|} \hline
November 30th, 2002 & Created fileformats.tex from fileformats.txt. \\ \hline
\end{tabular}
\end{table}

\end{document}



