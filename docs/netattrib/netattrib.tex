% $Id$
%
% gEDA - GPL Electronic Design Automation
% Copyright (C) 2003 Ales Hvezda
%
% This program is free software; you can redistribute it and/or modify
% it under the terms of the GNU General Public License as published by
% the Free Software Foundation; either version 2 of the License, or
% (at your option) any later version.
%
% This program is distributed in the hope that it will be useful,
% but WITHOUT ANY WARRANTY; without even the implied warranty of
% MERCHANTABILITY or FITNESS FOR A PARTICULAR PURPOSE.  See the
% GNU General Public License for more details.
%
% You should have received a copy of the GNU General Public License
% along with this program; if not, write to the Free Software
% Foundation, Inc., 675 Mass Ave, Cambridge, MA 02139, USA.

\documentclass{article}
%\usepackage{epsfig}
\usepackage{graphicx}
% This line will enable hyperlinks in the PDF output
% file.
\usepackage[ps2pdf,breaklinks=true,colorlinks=true]{hyperref}

\setlength{\parindent}{0pt}
\setlength{\parskip}{1ex plus 0.5ex minus 0.2ex}

\title{{\tt net=} attribute mini-HOWTO}
\author{Ales Hvezda\\
        \\
        This document is released under GFDL\\ 
        (\url{http://www.gnu.org/copyleft/fdl.html})}
\date{October 2nd, 2003}

\begin{document}

\maketitle
\newpage

\tableofcontents
\newpage

The information in this document is current as of 19991011.

\section{What is the {\tt net=} attribute used for?}
The {\tt net=} attribute is used to specify power, ground, and/or arbitrary
nets in the gEDA system.

The {\tt net=} attribute is used instead some of the other systems of
specifying power/ground (such as having power/ground pins on symbols
or power boxes).  Some devices have lots of power/ground pins and
having all of these pins on the symbol would increase its size and
make it unmanageable.  The {\tt net=} attribute is the power/ground
specification of choice in the gEDA system because of its simplicity
and versatility.

Now having said all this, you can have power/ground pins on a symbol,
but gnetlist will probably not recognize these nets connected to these
pins as separate power/ground nets.  Please keep this in mind as you
draw symbols.



\section{What is the format of the {\tt net=} attribute?}
Attributes in gEDA are simple text items which are in the form {\tt
name=value}.  All proper attributes follow this form.  Attribute names
are always lower case, but the value can be upper or lower case.
gnetlist and friends are case sensitive.  Typically net/signal names
by default are upper case.

Attribute can be attached to an object or in certain cases (like the
{\tt net=} attribute) can be free floating (not attached to anything).  The
free floating attributes are also called toplevel attributes.

The {\tt net=} attribute is a text item which takes on the following
form:

{\tt net=signalname:pinname,pinname,pinname,...} 
where
\\
\begin{tabular}{l p{4in}}
  {\tt net=}  &	The attribute name (always the same, lowercase) \\
  signalname  & The signal or net being defines (like +5V, GND,
  etc...) \\
  pinname     &	The pin name (or number) which is assigned to this
			signal/net (or pin names/numbers) \\
\end{tabular}

The signalname cannot contain the : character (since it's a
delimiter).  The pinname is the pin name (A1, P2, D1, etc...)  or pin
number (1, 2, 5, 13, etc...).  The pinname cannot contain the ,
character (since it's also a delimiter). pinnames are typically the
same sort of numbers/names like the pin\#=\# attribute (if your familiar
with that attribute).

You can only have ONE signalname per {\tt net=} attribute, but you can have
as many pinnames/numbers as you want.



\section{How do you actually use the {\tt net=} attribute}
You can place the {\tt net=} attribute in several places.  
Here's the list so far:
\begin{itemize}
\item Inside a symbol either as an attached attribute or an unattached
attribute (toplevel attribute).  Example which creates power/gnd nets:
{\tt net=GND:7 or net=+5V:14}

\item Outside a symbol (which is instantiated on a schematic) attached
as an attribute to override an existing {\tt net=} created
net/signal. Suppose a symbol has a {\tt net=GND:7} inside it already;
attaching this to the symbol: {\tt net=AGND:7}
overrides the GND net (on pin 7) calling it AGND and
connects/associates it to pin 7.

\item Outside or inside a symbol to connect a net to a visible pin
automatically.  This is still untested and still might have some
undesirable (negative) side effects.  Use with caution.

\item Attached to one of those special power/gnd symbol (like
vcc/gnd/vdd) and you can change what that symbol represents.  You
could change the ground symbol to create a net called DIGITAL\_GND
without editing the symbol (net=DIGITAL\_GND:1).

In the current symbol (19991011) library there are symbols named
{\tt vdd-1.sym}, {\tt vcc-1.sym}, {\tt vee-1.sym}, etc... which do not
have a {\tt net=} attribute inside, so you must attach the {\tt net=}
attribute yourself (in the schematic).

There also symbols named 5V-minus-1.sym, 12V-plus-1.sym, 9V-plus-1.sym
etc... which have the appropriate {\tt net=} attribute in them already
(can be overridden though).  You can use these symbol as examples of
how to use the {\tt net=} attribute.

\end{itemize}
You can have as many {\tt net=} attributes as you want.  Just remember that
{\tt net=} attributes attached to the outside of a symbol override any
equivalent internal (inside the symbol) {\tt net=} attributes.  If you run
into a case where this doesn't work, please let {\tt ahvezdaATgeda.seul.org}.
In fact, send any bug reports to that individual.



\section{Caveats / Bugs}
The {\tt net=} attribute/mechanism is fairly new, so there are bound
to be bugs (many bugs).  Here are some of the identified issues:
\begin{itemize}
\item As of 19991011 almost all of the symbols in the standard library
do not have the {\tt net=} attribute or any other power/ground
specifiers.  Hopefully this will be updated sometime (any
volunteers?).

\item Attach a special power symbol (vcc/gnd) to a already named net
will alias (rename) that net to the signalname specified in the {\tt
net=} attribute (in/attached to the vcc/gnd symbol).  You can override
this (so the reverse is true) by playing with the
"net-naming-priority".  Be careful with this.  There might be other
"aliasing" issues which have not been identified yet.

\item Creating a {\tt net=} attribute which associates a signal name
with a pin which is already visible on the symbol, is probably a bad
idea.  This does work, but all the ramifications have not been
explored yet.

\item It is probably a bad idea to have the same {\tt net=} attribute
attached several times.  Ales has not formalized what happens in this
case.  Just remember that the {\tt net=} attribute on the outside of a
symbol should override the internal one.
\end{itemize}

\section{Example}
Here's a schematic which uses standard symbols (note: the 7400 does
not have the {\tt net=} attribute inside yet).  This schematic
consists of a 7400 with the {\tt net=} attributes attached for power
and ground, One of the input pins grounded using a gnd symbol and the
other at a logic one using the vcc symbol (with an attached {\tt net=}
attribute).  One of the input net is named, but as you will see, the
netname is replaced by the {\tt net=} signal name (see above for more
info on this).  The output is pulled up with a pull up resistor which
has power specified using the +5V symbol.

	[ Maybe insert picture of schematic as well ]
	
\begin{verbatim}
v 19991011
C 38700 58100 1 0 0 7400-1.sym
{
T 39000 59000 5 10 1 1 0
uref=U100
T 38900 59500 5 10 1 1 0
net=GND:7
T 38900 59300 5 10 1 1 0
net=+5V:14
}
N 38700 58800 37400 58800 4
{
T 37600 58900 5 10 1 1 0
netname=NETLABEL
}
N 37400 58800 37400 59200 4
N 38700 58400 37400 58400 4
N 37400 58000 37400 58400 4
C 37300 57700 1 0 0 gnd-1.sym
C 37200 59200 1 0 0 vcc-1.sym
{
T 36800 59200 5 10 1 1 0
net=+5V:1
}
N 40000 58600 41600 58600 4
{
T 41200 58700 5 10 1 1 0
netname=OUTPUT
}
C 40700 58800 1 90 0 resistor-1.sym
{
T 40800 59200 5 10 1 1 0
uref=R1
}
N 40600 58800 40600 58600 4
N 40600 59900 40600 59700 4
C 40400 59900 1 0 0 5V-plus-1.sym
\end{verbatim}

	gnetlist (using the geda netlist format) run using this
sample schematic outputs this:

\begin{verbatim}
START header

gEDA's netlist format
Created specifically for testing of gnetlist

END header

START components

R1 device=RESISTOR
U100 device=7400

END components

START renamed-nets

NETLABEL -> +5V

END renamed-nets

START nets

+5V : R1 2, U100 14, U100 1 
GND : U100 7, U100 2 
OUTPUT : R1 1, U100 3 

END nets
\end{verbatim}
Notice how NETLABEL was renamed (aliased to the +5V net).  	



\section{Final notes}
Send all bugs to {\tt ahvezdaATgeda.seul.org} or {\tt
  geda-devATgeda.seul.org} (mailing list, please subscribe first) 

[I'm sure there's more to say here]

\end{document}
