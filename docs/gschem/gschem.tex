% $Id$
%
% gEDA - GPL Electronic Design Automation
% Copyright (C) 2003 Ales Hvezda
%
% This program is free software; you can redistribute it and/or modify
% it under the terms of the GNU General Public License as published by
% the Free Software Foundation; either version 2 of the License, or
% (at your option) any later version.
%
% This program is distributed in the hope that it will be useful,
% but WITHOUT ANY WARRANTY; without even the implied warranty of
% MERCHANTABILITY or FITNESS FOR A PARTICULAR PURPOSE.  See the
% GNU General Public License for more details.
%
% You should have received a copy of the GNU General Public License
% along with this program; if not, write to the Free Software
% Foundation, Inc., 675 Mass Ave, Cambridge, MA 02139, USA.

\documentclass{article}
%\usepackage{epsfig}
\usepackage{graphicx}
% This line will enable hyperlinks in the PDF output
% file.
\usepackage[ps2pdf,breaklinks=true,colorlinks=true]{hyperref}

\setlength{\parindent}{0pt}
\setlength{\parskip}{1ex plus 0.5ex minus 0.2ex}

\title{gEDA gschem Users Guide}
\author{Ales Hvezda\\
        \\
        This document is released under GFDL\\ 
        (\url{http://www.gnu.org/copyleft/fdl.html})}
\date{September 21st, 2003}

\begin{document}

\maketitle
\newpage

\tableofcontents
\newpage


\section{Overview}

\section{Introduction}
This is a very basic and crude first attempt at getting some real user
docs written for {\tt gschem}.  This document is hardly finished or polished
but it is better than having no docs at all (which was the case for
the longest time).  Please forgive the tone of this document; I (Ales)
am writing this just to get something done.  I am by no means a
technical writer so I am sure some of my sentences are truly awful.
As everybody says, programmers should not write documentation for
their programs and this document is a perfect illustration of a
programmer breaking that rule.  Please report any
errors/inconsistencies to {\tt ahvezda@geda.seul.org}.

This document assumes you understand basic schematic capture
concepts.  For example that a component represents something and that 
nets and buses interconnect these components to form a schematic
etc...  Eventually this document will describe all this in more
detail, but for now this document is will describe how schematic
capture is accomplished in the gEDA system.

\section{Overview}
{\tt gschem} is the schematic capture program in the gEDA tool suite.
It's sole purpose is facilitate the input of circuit schematics or
block diagrams graphically.  {\tt gschem} is also the symbol editor for the
gEDA system.  This is possible because the schematic format and symbol 
format are exactly the same.

\section{Running {\tt gschem}}
Running {\tt gschem} is straightforward once you have installed it
on your GNU/Linux or UNIX system.

The first step is to build and install the gEDA tools.
This step is not covered in this document.  You can tell if you have
installed the tools correctly by running the following commands:
\begin{verbatim}

        libgeda-config --version
        gesym-config --version
        which gschem
        ldd `which gschem`

\end{verbatim}
The first two should return the version of the installed tools
(libgeda and the symbol library) and the next command should return
the path to the {\tt gschem} binary.    The final command (only on
unix-like operating systems which include the {\tt ldd} utility for
listing dynamic dependencies of executibles or shared objects)
will return which libraries are linked to {\tt gschem}; all of the
request libraries must be found for {\tt gschem} to run.  If these commands
do not return the expected results, then most likely the gEDA tools are
not installed properly.  Please see the appropriate INSTALL docs (which
came with the distribution) for more info on installing the gEDA tools.


To run gschem execute ``{\tt gschem}'' at any shell prompt (without
the quote marks).  This will
start the gschem program with a blank schematic.  To look at an
example schematic which came with gschem run ``{\tt gschem
  schematic\_name.sch}'' For a listing of the various command line flags
run ``{\tt gschem -h}''.  For a detailed explanation of the command line
flags look at the gschem man page (``{\tt man gschem}'').

\section{User interface}
There are several ways to interact with gschem.  gschem
requires a keyboard and mouse.  There are three ways to initiate an 
operation or command:
\begin{itemize}
\item Using the mouse to select the operation off a menu 
\item Typing the keyboard shortcut(s).
\item Draw the appropriate stroke (if stroke support is enabled)
\end{itemize}
To make usage matters more confusing, selecting an operation
off of the menus behaves slightly differently than typing the
keyboard shortcut.  Most of the operations operate on the currently
selected object(s), hence you need to select the object first before
manipulating them.  The menu selected operations usually require some
more input (usually a mouse click) after they are picked off of the menu.
The keyboard shortcut operations take that required input as the current
mouse position.  This saves an extra click since you can position
the mouse at the right place, type in the shortcut(s), and the command
then executes.  Note, you can change this so that both menu and shortcut
behavior is exactly the same.  See the section on the resource file for
more info on how to configure this.

Most of the interaction with gschem is fairly mode oriented
(similar to the great text editor vi).  If you select operations off of
the menu, then you are placed into the corresponding mode (like copy
or move mode).  You must then select an anchor point (or whatever the
appropriate point is) to continue the operation.  Most of the commands
off of the menu expect the objects to be already selected.  Some of the
modes persist after being execute while other immediately return you
into select mode (the default mode).

The shortcuts are also mode like in nature.  Most of the default
shortcuts are for the various commands are not single keystrokes.
There are a few which are single keystrokes (like zoom in: `z' or
pan: `x'), but most are typically two keystrokes long.  As examples,
to execute File/Save you would type `f' and `s' (without the quotes)
or Add/Line is `a' and `l'.  You can get a listing of the shortcuts by
picking Help/Hotkeys.  You can also see the hotkey assignments in the
pulldown menus as well.  The shortcuts are defined in the resource files
({\tt system-gschemrc}, {\tt ~/.gEDA/gschemrc}, or {\tt
  `pwd`/gschemrc}).  See the section 
on the resource file for more info.

The mouse button actions in gschem are mostly configurable.
The first mouse button is always used to select objects or pick points.
This button is not configurable.  The second mouse button is either a
copy/move action (when held down over an object), a repeat last command
or used to draw a stroke to execute a command.  The third mouse button is
either a mouse pan (when held down as the mouse is moved) or a popup menu.
The behavior of the second and third mouse buttons is controlled through
the resource file (see the section below for more info).

\section{Basic Operations}
There are nine basic kinds of operations in gschem:
\begin{itemize}
\item File related (Open/Close/Save/Print etc...)
\item Addition of objects (Add nets/lines/pins/components etc...)
\item Manipulate objects (move/copy/delete/rotate/mirror etc...)
\item View operations (Zoom/Pan/Redraw etc...)
\item Hierarchy operations (Down Schematic/Down Symbol/Up)
\item Attribute operations (Attach/Detach/Toggle visibility etc...)
\item Buffer operations (Copy / Cut / Paste)
\item Option changing (text size/gridding/snap etc...)
\item Editing Grips
\end{itemize}
This section will describe each of these.
The terms ``page'' and ``schematic'' both refer to the same thing (a
collection of objects that are loaded and being displayed).
A window is that big window in which schematics are displayed
and manipulated.

\subsection{File related operations:}

\begin{itemize}
\item {\bf New Window} Open a new window.  Each window is totally
  separate from all other windows.
  
\item {\bf New Page} Open a new page.  Usually this page will be
  called untitled\_N.sch, where N is an incrementing number.
  
\item {\bf Open Page...}  Open an page from disk.  This will pop up a
  dialog box.  Please see the later section on how to use the more
  advance features of this dialog box.
  
\item {\bf Close Page} Close the currently displayed page.  This will
  prompt you to save if you have modified the page.

  
\item {\bf Revert Page} Close and reopen the currently displayed page.
  This will not prompt you to save the current page, but will quickly
  discard any changes you have made and reopen the saved schematic
  from disk.  Use with caution.
        
\item {\bf Save Page} Save the page current page.  If the page is
  called untitled\_N.sch (where N is a integer), then a Save As...
  dialog box will appear.
  
\item {\bf Save Page as...}  Opens the Save As.. dialog box.  Clicking
  Save As will save the currently displayed schematic.  The displayed
  schematic will be called the new specified name.
  
\item {\bf Save All} Unconditionally saves all schematics loaded in
  memory.
  
\item {\bf Print...}  Brings up the Print dialog box.  Please see the
  dialog box section for more info on this.
  
\item {\bf Write PNG...}  Brings up the Write PNG dialog box.  Please
  see the dialog box section for more info on this.  Note you must
  have libgdgeda install (and any required dependencies) if you want
  to output images.
  
\item {\bf Close Window} Closes the current window.  If there are any
  modified schematics, a "Are you sure" dialog box will appear.
  Clicking OK will cause all unsaved schematics to be lost.
  
\item {\bf Quit} Like Close window, but close all opened window.  A
  "Are you sure" dialog box will appear for each window which has
  unsaved schematics
  
\item {\bf Page Manager} Brings up a dialog box which lets you pick
  pages as well as see various info about the pages loaded in memory.
  Please see the dialog box section for more info on this.
  

\item {\bf Page Next}\\
  {\bf Page Prev} These two options allow you to move in between
  opened pages.
  
\item {\bf Page Discard} Closes the current schematic forcefully.  If
  the current schematic is NOT saved then the data is lost forever.
  You will NOT be prompted or warned if you have modified the current
  displayed schematic.  Use with caution.  Useful if you want to
  unload a schematic quickly.

\end{itemize}

\subsection{Addition of objects:}
An object is one of the following: line, circle, pin, net, box, bus,
text/attribute, and component.  A component is a collection of lines,
circles, pins, boxes, text, and attributes.
\begin{itemize}
\item {\bf Add Component} Opens up a dialog box which lets you place
  components from the component libraries.  To place a component do
  the following:
\begin{enumerate}
\item Select a Component library (which, btw, is specified in the
  Resource files, see section below) from the left list

\item Select a Component from the right list.
  
\item Move the mouse into the main drawing window (you should see an
  outline).
  
\item Press the first mouse button to place the component.
  
\item Keep pressing the first mouse button to place additional
  instances of the component.

\end{enumerate}

If a component name is already selected, hitting apply and moving the
mouse into the main window will allow you to place that component
again.

You can rotate the component before you place it by clicking the
middle button.  For every button click, the component will be rotate
90 degrees.

To cancel a component place press the last mouse button or the ESC
key.

Please see the dialog box section for detailed information on this
dialog box.

\item {\bf Add Net} Draws a net segment.  If you select this off of
  the top menu bar then:
\begin{enumerate}
\item Press the first mouse button to start the net.
\item Press the first mouse button to end the net.
\item Another net will start at the last endpoint.
\item Press the first mouse button to end that net etc...
\end{enumerate}
Press the last mouse button or ESC to cancel any net in progress.

If you started this using the keyboard shortcut then the net start
immediately at the last mouse position and then it behave exactly as
above (except for line \#1).

After drawing a net segment (or segments) and canceling the last net
segment, you are automatically placed in select mode.  You must pick
add net again or type the shortcut to add more nets.

You can hold down the control key to draw non-orthogonal nets.  Just
keep in mind that you cannot connect anything to the middle of a
non-orthogonal net.

The boxes at the end of the nets are connectivity cues.  Filled boxes
signified a dangling net (not connected to anything).  Filled circles
are midpoint connections/junctions.  These cues are drawn
automatically and are an indicator of electrical connectivity.

See the section on electrical connectivity below for more information.

\item {\bf Add Bus} Basically the same thing as nets, except that it
  draws buses.  Buses are very new and there are many aspects which
  are not defined yet, so keep that in mind as you uses buses.  *More
  to be added here eventually*
  
\item {\bf Add Attribute...}  Brings up the add attribute dialog box.
  This dialog box is ONLY used to add attributes.  It does not display
  or manipulate already placed attributes.
  
  An attribute is nothing more than a text item which is in the form
  {\tt name=value} (there cannot be any spaced to the left or right of the
  name,value pair).  It can be either unattached or attached.

 To add an unattached attribute do the following:

 \begin{enumerate}
 \item Select an attribute name off of the pulldown list --or-- type
   in the attribute name into the name entry.
 \item Type in a value for the attribute.
 \item Pick any of the attribute options.
 \item Click Apply and the attribute will be placed.
 \end{enumerate}
 
 If you want to attach an attribute to an object then select the
 desired object first and then press the add attribute apply button.
 The text should be yellow which signifies an attached attribute.  If
 you click on an object which has attached attributes, the attached
 attributes should be selected as well.
 
 If you select Add/Attribute... off of the pull down menus then you do
 not have much control as to where the attribute gets placed (it gets
 places either at the lower left hand corner of the object extents or
 at the origin of any selected object).  However, if you execute
 Add/Attribute using the hot key then the current mouse position is
 used as the anchor point for the attribute item.
                
 You cannot place an incomplete attribute (an attribute without a name
 and value).
 
 Please see the section on attributes below for more info on how to
 use attributes and more details.
 
\item {\bf Add Text} Opens up the text add dialog box.  To place text:
\begin{enumerate}
\item Type in text in entry field
\item Hit enter or click Apply
\item Move the mouse into the main window (an outline of the text
  should appear and follow the mouse)
\item Press the first mouse button to place the text
\end{enumerate}
If you leave the text add dialog box open you can place the same text
item again and again by just clicking apply (or pressing enter) and
moving the mouse into the main window.

Text which is placed will be automatically capitalized.  Please see
the Resource file section below on how to control this behavior.

To cancel a text place press the last mouse button or the ESC key.

If you create text in the form {\tt name=value}, then you are creating
attributes.  gEDA allows for general attributes to be free floating
(or unattached).  It is a good idea to change the color of these
floating attributes to the current attribute color (which is also
called the attached attribute color) to signify that this text item is
an attribute.

You can rotate the text before you place it by clicking the middle
button.  For every button click, the text will be rotate 90 degrees.

\item {\bf Add Line} Draws a line in the same fashion as drawing nets
  with the following exceptions:
\begin{itemize}
\item A line has no electrical significance
\item Only a single line is drawn
\item You keep drawing lines as long as you are in line drawing mode.
\end{itemize}
To cancel a line rubberband in progress press the last mouse button or
the ESC key.
        
\item {\bf Add Circle} To draw a circle (picking Add Circle off of a
  menu):
\begin{enumerate}
\item Pick the center of the circle with the first button
\item Move the mouse to see an outline of the circle
\item Press the first mouse button to finalize the circle
\end{enumerate}

To draw a circle (typing the shortcut), same as above except that the
center of the circle is picked for you at the last mouse position when
you type the shortcut.

To cancel a circle rubberband in progress press the last mouse button
or the ESC key.

\item {\bf Add Arc} To draw an arc (picking Add Arc off of a menu):
\begin{enumerate}
\item Pick the center of the arc with first mouse button
\item Move the mouse and pick the next point.  The rubberbanded line
  represents the radius
\item A dialog box will appear: specify the start angle (in degrees)
  and specify the degrees of sweep.
\item Press OK to finalize the values.
\end{enumerate}
The start angle can be positive or negative.  The degrees are
specified using the standard Cartesian coordinate system.  The degrees
of sweep can be positive or negative.

To cancel an arc in progress (while rubberband the radius) press the
last mouse button or the ESC key or press the Cancel button in the arc
dialog box.

\item {\bf Add Pin} Draws a pin in the same fashion as drawing nets
  with the following exception:
\begin{itemize}  
\item You keep drawing pins as long as you are in pins drawing mode.
  \end{itemize}
  To cancel a pin rubberband in progress press the last mouse button
  or the ESC key.
\end{itemize}

\subsection{Manipulate objects:}
Most of these operations are modes which operate on the current
selected objects.  Before you can execute the desired function you
must select the objects you want to rotate.

\begin{itemize}
\item {\bf Undo/Redo} Undo does exactly that, it undos the last action
  which changed the schematic.
  
  Here is how gschem implements undo: Basically after every action
  (including zooming and panning) the schematic is saved to disk (in
  /tmp).  Gschem do clean up after itself when you exit.  Should
  gschem crash, the saved files are left alone in /tmp.
  
  Redo only applies when you do an Undo.  You can undo something and
  then immediately redo it.  However if you do anything in between you
  will lose the undo info.  You can undo and redo to your hearts
  desire up and down till you reach the max undo levels.
  
  You can change the behavior of undo and redo by changing the
  defaults in one of the *gschemrc files.  Please see the
  system-gschemrc for more info.
  
\item {\bf Select Mode} Select mode is the initial mode gschem starts
  up in.
  
  In select mode you can select objects.
  
  You can pick single objects by clicking on them.  If an object is
  already selected than clicking on that object will keep it selected.
  If multiple objects overlap then clicking in one spot will cycle
  through the objects.
  
  If you hold down the SHIFT key and click, you can select and
  deselect multiple objects.  Doing this with multiple overlapping
  objects will cause the selection to cycle among the possible object
  selections.
  
  If you hold down the CONTROL key and click, you will toggle the
  object in and out of the current selection list.
  
  To select multiple objects press and hold the first mouse button and
  drag the mouse till a selection box appears.  Drag the mouse
  (selection box) till it encompasses the objects and release the
  first mouse button.  The objects must be completely encompassed.
  Using the selection box takes some practice.  If any previous
  objects were selected, they will be unselected.
  
  If you hold down the SHIFT key while drawing a selection box then
  you will add to the currently selected objects.  Objects cannot be
  removed using the selection box and holding down the SHIFT key.
  
  If you hold down the CONTROL key while drawing a selection box then
  you will toggle any encompassed objects.  If an object was selected
  then it will be unselected and vice versa.
  
  If you pick an object which has attributes which are attached to it,
  then these attributes will be selected as well.  If you just want to
  select the object, you must deselect the attributes.  Invisible
  attached attributes are also selected when you pick the object This
  behavior is handy if you want to manipulate/change an object and all
  of its attributes (moving/copying the object around).
  
  The selection mechanisms are not obvious and do require some
  practice.  There are some quirks so please report them as you come
  across them.
  
\item {\bf Edit...}  Allows you to edit: Text - brings up a dialog box
  which allows you to edit the text string and size
  
  Attributes - a much more sophisticated version of above specifically
  for attributes.
  
  All attributes attached to a component - brings up a multi-attribute
  editing dialog box.

\begin{enumerate}
\item Select the item (text/component) to edit.
\item Pick or type the shortcut for Edit/Edit...
\item Make the appropriate change and press OK
\end{enumerate}
Note, only the item will be edited; bulk text/component editing is not
supported (yet).

\item {\bf Edit Text...}  Allows you to edit text only, regardless if
  that text is an attribute or just a plain text string.
\begin{enumerate}
\item Select the text item text to edit.
\item Pick or type the shortcut for Edit/Edit Text...
\item Make the appropriate change and press OK
\end{enumerate}
Note, only the item will be edited; bulk text editing is not supported
(yet).

\item {\bf Copy Mode} This mode allows you to copy the currently
  selected objects.
  
  To copy objects (picking Edit/Copy Mode off of a menu):
\begin{enumerate}
\item Select the desired object(s)
\item Pick Edit/Copy Mode off of a menu
\item Pick an origin point for the copy (first button)
\item Move the outline to the desired position
\item Pick the destination point (first button)
\end{enumerate}

To copy objects using the shortcut for copy mode is almost the same as
above except that the origin point is selected automatically for you
once you hit the copy mode shortcut.

After finishing the copy, you are automatically put back into to
select mode.

Holding down the CONTROL key as you move the outline around will
constrain the movement to be either horizontal or vertical.

\item {\bf Move Mode} Move mode is just like copy (above) except that
  instead of copying you are actually moving the objects around.
  
\item {\bf Delete} Delete allows you to remove objects off of the
  page.
  
  To delete objects:
  \begin{enumerate}
  \item Pick the desired object(s).
  \item Select or type the shortcut for Edit/Delete
  \item Objects will be deleted immediately  
  \end{enumerate}
  Use this with caution since there is NO undo yet!       
  
\item {\bf Rotate 90 Mode} Rotate mode allows you to rotate objects 90
  degrees around a pivot/center point.
  
  To rotate objects (picking Edit/Rotate 90 Mode off of a menu):
\begin{enumerate}
\item Pick the desired object(s).
\item Select Edit/Rotate 90 Mode off of a menu
\item Pick the pivot or center point of the rotate
\end{enumerate}
Rotating objects using the shortcut is similar to above except that
the center point is the last mouse position at which you typed the
shortcut.
                
The objects will be rotate 90 counter clockwise.  You can keep rotate
objects 90 degrees till you have them in the wanted rotation.  Text
will always appear upright.
                
\item {\bf Mirror Mode} Mirror mode allows you to mirror objects
  horizontally around pivot.
  
  To rotate objects (picking Edit/Rotate 90 Mode off of a menu):
\begin{enumerate}
\item Pick the desired object(s).
\item Select Edit/Mirror Mode off of a menu
\item Pick the pivot point of the mirror
\end{enumerate}
Mirroring objects using the shortcut is similar to above except that
the pivot point is the last mouse position at which you typed the
shortcut.

Objects are mirrored horizontally about the pivot point.  If you want
to get a vertical mirror then rotate and mirror the object(s) till you
get the desired position.

Mirroring of embedded components is not supported.

\item {\bf Slot...}  Slot... allows you to change the slot of a
  multiple slot component.  The component must support slotting (see
  the Components/Symbols section for more info on this).
  
  To change the slot:
\begin{enumerate}
\item Select the desired component.
\item Pick or type the shortcut for Edit/Slot...
\item Type in the new slot number
\item Press OK.
\end{enumerate}                        
Slotting is still in development so expect some quirks.
        
\item {\bf Color...}  This option allows you to change the color of
  any selected object (with the exception of components).
  
  To change the color of the currently selected objects:
\begin{enumerate}
\item Select the desired object(s).
\item Pick or type the shortcut for Edit/Color...
\item A dialog box with a drop down menu will appear.
\item Pick the new color
\item Press Apply.
\end{enumerate}
The color change will only take effect once you press Apply.  The
dialog box will not change to the color of the current object(s) (ie
it only lets you change the color and does not get the current color
of selected object(s)).

You can leave this dialog box up and select new objects and change
their color by pressing Apply.

\item {\bf Lock}\\
  {\bf Unlock} Lock and unlock allow you to lock/unlock components in
  a schematic.  Locking means that a component cannot be selected by a
  single click.  This is useful for title blocks and other components
  which should not be selectable because you might have other objects
  inside its boundaries and clicking inside the component (titleblock)
  would be a distraction.
  
  To lock/unlock components:
\begin{enumerate}
\item Select the desired object(s)
\item Pick or type the hot key for lock/unlock
\end{enumerate}                        
The components will not be selectable with a single click.  To select
a locked component, use a selection box to select it.  This is the
only way to select a locked component.

You can lock and unlock regular objects (lines/pins/boxes...)  which
is nice when you are drawing something and an object is in the way
continuously.  Just lock it and you will not have to think about it
when you click to select other objects.  However, locking an object is
not preserved in the file format so once you quit any locked objects
(lines/pins/boxes...) will become unlocked.  The locked/unlocked state
of components is preserved in the file format though.

\item {\bf Line Width \& Type...}  This dialog box lets you control
  the width and type of lines, boxes, circles, and arcs.  This dialog
  box is still under development.
  
  Select an object and then select this option will bring up a dialog
  box which lets you set the line width, line type, line dash length,
  and line dash spacing.  Pressing OK will apply the changes.
  
\item {\bf Symbol Translate...}  Symbol translate is used to take
  whatever is drawn and translate it around using a inputed value (in
  mils).  This operation is mainly used to translate symbols around
  (mainly to the origin).
  
  To translate a symbol, just select or type the shortcut for
  Edit/Symbol Translate... and a dialog box will appear.  Enter the
  amount you want translate (positive or negative) in the X and Y
  directions (the same value will be applied to both directions).
  
  If you enter a 0, then all the objects will be translated to the
  origin.  If you are drawing a symbol, zoom in a bit and then execute
  this (0 for the amount), since there are still some bugs in this
  operation.  It is a requirement that snap be on and that the grid
  snap spacing is set 100 mils when creating symbols.
  
\item {\bf Embed Component} \\ {\bf Unembed Component} gschem supports
  a concept of embedded components which components in a schematics
  which do not require an external symbol file of any sort.  All the
  information necessary to display a component is placed in the
  schematic file.  This causes schematic files to be significantly
  larger, but it makes it easy to share schematics with other people
  or archive schematics away, since you have any dependencies on
  symbol files.  You should only embed components when absolutely
  necessary.
  
  Embed Component and Unembed component work exactly alike:
\begin{enumerate}
\item Select any components you wish to (un)embed
\item Pick or type the shortcut for Edit/(Un)Embed Component
\end{enumerate}
If you want the unembed a component you must have a symbol with the
same name in the component library search path (other wise the unembed
will not work).

You can only embed and unembed components.  Also, you cannot embed and
then mirror a component (this is a limitation of gschem and will
eventually be fixed).

You can also place embedded components directly in the Add
Component... dialog box.

\item {\bf Show Hidden Text} This operation makes all hidden/invisible
  text visible.  To use it, simply pick the option (or type its
  shortcut) and all the invisible text will appear.  Visible text is
  unaffected by this operation.
  
  This operation is useful when drawing/debugging symbols.
                
  When hidden text is visible, the text "Show Hidden" will appear in
  the lower, righthand corner.

\end{itemize}

\subsection{View operations (Zoom/Pan/Redraw etc...)}

\begin{itemize}
\item {\bf Redraw} This option redraws the current display.  This is
  useful when you have mouse/component/line/text etc... droppings left
  over from a previous action.  It is also useful when you want to
  update all visual connectivity cues.
  
\item {\bf Pan} Pan lets you move around the display.
  
  To pan the display (picking View/Pan off of a menu):
\begin{enumerate}
\item Select View/Pan off the menu
\item Click the first mouse button at the new center of the display.
\end{enumerate}  
  To pan the display using the shortcut is much simpler, simply type
  the shortcut and the display will pan the display to the current
  mouse location.
  
  You can also enable mouse panning if you add:
  
  {\tt (third-button "mousepan")}
  
  to one of the gschemrc files.  See the section on the resource file
  for more info.  Mouse panning is very nice for small schematics, but
  tends to lag on larger ones.
  
\item {\bf Zoom box} Zoom box allows you to specify a zoom window for
  zooming in.
        
  To use the zoom box (picking View/Zoom box off of a menu):
\begin{enumerate}
\item Select View/Zoom box off the menu
\item Click and hold the first mouse button
\item Drag the mouse drawing the zoom box around the area you want to
  zoom
\item Release the mouse button and the display will zoom
\end{enumerate}
To use the zoom box by typing the shortcut is similar.  Once you type
the shortcut, the zoom box will start immediately using the current
mouse location as the first corner of the box.

Zoom box will attempt to zoom the requested area, but some boxes are
not legal and gschem will do it's best to zoom the requested area.

\item {\bf Zoom limits} Zoom limits will zoom the display attempting
  to fit all the placed objects on the screen.
  
  Simply pick View/Zoom limits off of a menu or type its shortcut and
  the display will be redrawn.
  
  There are the special cases (like a single horizontal line) which
  sometimes do not display correctly (or as expected).  Hopefully
  these exceptions will be fixed someday.
  
\item {\bf Zoom In} \\
  {\bf Zoom out} Zoom In/Out zoom the display using the picked mouse
  location as the center of the display (or the current mouse location
  if using the shortcut).  These commands always zoom in/out by a
  factor.
  
  To use Zoom In/Out (picking View/Zoom box off of a menu):
\begin{enumerate}
\item Pick Zoom In/Out off of the menu
\item Click the first mouse button as the center of the zoom
\end{enumerate}        
Using the shortcut is similar except that the current mouse position
serves as the center of the new display.

\item {\bf Zoom full} Zoom full will zoom the display to the maximum
  possible displayable view.
  
  Simple select it off of a menu or type it's shortcut and the display
  will be zoomed.

\end{itemize}

\subsection{Hierarchy operations (...)}

\begin{itemize}
\item {\bf Down Schematic} Go down into a symbol, opening up any
  underlying schematics.  Basically this will open up an underlying
  schematic of the selected component if it exists in the source
  library search path.  See the Resource File section on how to define
  this path.
  
  There are currently two ways of specifying that a symbol has an
  underlying schematic or schematics:
  
\begin{enumerate}
\item The underlying schematic must have the same name as the symbol
  but have a .sch extension and must follow the \_\# suffix naming
  convention.  See the Files section below on this convention.
  
\item Attach an attribute to the symbol called {\tt source=filename.sch}
  filename.sch is not a path to the symbol, but rather the basename
  (last file in the path specifier) of the symbol path.  The
  underlying schematic will still be searched in the source-library
  path.  You can specify multiple {\tt source=} attributes.  The underlying
  schematics will be opened in the order that the {\tt source=} attribute is
  found.
\end{enumerate}
  If there multiple underlying schematics, they will be loaded.
  Movement between the schematic pages is restricted (to the same
  level of the same set of underlying schematics) unless the rc
  keyword enforce-hierarchy is modified to allow for a freer hierarchy
  traversal mode.  See the Resource File section for more info.
  
  It is also recommend that you maintain unique names for the various
  levels (when using the {\tt source=} attribute) to avoid possible
  confusion.  The hierarchy mechanisms are fairly new so expect some
  odd behavior (and please report it)
  
\item {\bf Down Symbol} This option will open up the symbol of the
  selected component.
  
  Once the symbol is open, the user can edit it and save it.
  
  At this time, the toplevel schematic will not see the symbol change
  unless the toplevel schematic is reloaded or File/Revert is
  executed.  This will be fixed eventually.
  
\item {\bf Up} This option will move up the hierarchy (if there are
  pages above the currently displayed page).
  
\item {\bf Documentation} Open any documentation available for the
  selected symbol/component.
  
  The job is handed over to "gschemdoc", which makes a best-effort
  attempt of finding relevant documentation.
  
  The documention would normally be in PDF, HTML, text or image
  format, but gschemdoc tries to be as transparent as possible on this
  account.
  
  First and foremost, the attribute "{\tt documentation=}" is assumed to
  point to the documentation. This attribute should either be the
  filename (basename) of the document, or it should be a complete URL.
  
  If it is a filename, and the file is found locally (in
  /usr/share/gEDA/documentation or otherwise), the relevant viewer
  will be initiated. Otherwise, a Google search for the document will
  be initiated.
  
  If there is no documentation attribute, the attributes "device" and
  possibly "value" will be consulted in much the same way as for
  "documentation". File searches will be made in forms of filenames
  like "device-value.pdf" and "device.pdf".
  
  Failing that, the file name for the symbol itself will be used as
  basis for the search.

\end{itemize}

\subsection{Attribute operations (Attach/Detach/Toggle visibility etc...)}
An attribute is nothing more than a text item which is in the form
{\tt name=value}.  It can be either unattached or attached.

The operations in this group manipulate attributes only.
Most of these operations have no effect on plain text objects.

\begin{itemize}
\item {\bf Attach} The Attach command allows you to take a text item
  (in the proper form; {\tt name=value}) and attach it to another object.
  
  To use Attributes/Attach:
\begin{enumerate}                
\item Select the object which will receive the attributes
\item Select the text object(s) which will be attached to the above
  object
\item Pick or type the shortcut for Attributes/Attach
\end{enumerate}
The order of the sequence of selecting the object and then the text
items is important; gschem will not allow you to select the text items
first and then the object.  After going through the above sequence the
text item will turn yellow (or the current attached attribute color)
signifying that the text item is an attached attribute.

You cannot attach a single attribute to several different objects.
You cannot attach non-text items as attributes.

\item {\bf Detach} Detach allows you to deassociate attributes from
  objects.
  
  To deselect an object of all attributes:
\begin{enumerate}
\item Select the object of interest
\item Pick or type the shortcut for Attributes/Detach
\end{enumerate}                
All the attached attributes (even if they are not selected) will be
detached from the object.  This behavior is probably broken and will
eventually be fixed (so that only selected attributes are detached).

When you detach attributes then they turn red (or the current detached
attribute color).  This color changes allows you to spot text which
was an attribute and is now dangling (unattached).

\item {\bf Show Value}\\
  {\bf Show Name} \\
  {\bf Show Both} These operations allow you to control which part of
  the attribute string is visible.  Usually you are just interested in
  seeing the value of the attribute, but there are circumstances where
  seeing the name and value (or maybe just the name) would be useful.
  
  To use the options:
\begin{enumerate}
\item Select the attribute(s) of interest
\item Pick or type the shortcut for Attributes/Show *
\end{enumerate}        
The text item(s) should immediately change.

These operations only work on text items which are in the form
{\tt name=value}

\item {\bf Toggle Vis} This operation allows you to toggle the
  visibility of attributes.
  
  To use this option:
\begin{enumerate}
\item Select the text item(s) of interest
\item Pick or type the shortcut for Attributes/Toggle Vis
\end{enumerate}
The text item(s) should change their visibility immediately.
                
If you make an attached attribute invisible, then you can simply
select the parent object and select Toggle Vis and the attribute will
be come visible (likewise any visible attributes attached to that
object will become invisible).

If you make a free floating (unattached) attribute invisible, then the
only way to make it visible (and all other invisible attributes) is to
use the Edit/Show Hidden Text option.

\end{itemize}
\subsection{Buffer operations}
Gschem supports 5 copy/cut/paste buffers which are visible across all
opened pages and windows.

\begin{itemize}
\item {\bf Copy} To copy something into a buffer:
\begin{enumerate}
\item Select the objects you want to copy.
\item Select Buffer/Copy/Copy into buffer \#.
\end{enumerate}

\item {\bf Cut} Cut is like copy in that it removes the objects from
  the schematic
  
\item {\bf Paste} To paste a buffer into the current schematic:

\begin{enumerate}
\item Fill the buffer using the above Copy or Cut.
\item Go to the new schematic page/window.
\item Select Buffer/Paste/Paste from buffer \#.
\item Click the first mouse button to pick an anchor point.
\item Move the mouse to the final spot.
\item Click the first mouse button again.
\end{enumerate}

\end{itemize}

\subsection{Option changing (text size/gridding/snap etc...)}

\begin{itemize}
\item {\bf Text Size...}  This command pops up a dialog box which
  allows you to specify the text size of all text (including
  attributes placed with the Add/Attribute... dialog box).
  
  The text size is in points (1/72").  The default text size is 10
  point text.  The smallest text size is 2 points.
  
\item {\bf Toggle Grid} Toggles the visible grid
  
\item {\bf Toggle Snap} Toggles the snap.  Be very careful using this.
  Connections between pins and nets (and nets to nets) depends on
  being exactly connected.  Turning of the grid will almost guarantee
  that nets/pins do not connect.
  
  Before you translate a symbol using Edit/Symbol Translate, make sure
  the snap is on.
  
  When snap mode is off, the text "Snap Off" will appear in the lower,
  righthand corner.
  
\item {\bf Snap Grid Spacing...}  This option brings up a dialog box
  which allows you to change the snap grid spacing (not the grid
  spacing).  The units for this spacing are mils.
  
  Before you translate a symbol using Edit/Symbol Translate, make sure
  this spacing is set to 100.
  
\item {\bf Toggle Outline} Toggles between drawing the outline of the
  current selection or just drawing a box when doing
  moves/copies/component and text places.  The outline mode looks
  better, but tends to be significantly slower than using the box
  (bounding box) mode.
  
\item {\bf Show Log Window} This option displays the log window if it
  has been closed or disabled from being displayed when you start up
  gschem.
  
\item {\bf Show Coord Window} This option displays a dialog box which
  displays the current x, y location of the mouse pointer in screen
  (pixels) and world coordinates (mils)

\end{itemize}

\subsection{Grips}
Grips are a mechanism used in gschem to provide an easy way of
modifying objects inside schematics.  When you select an object,
little squares are placed in strategic locations (line end points or
circle radius point or corners of a box) which allow you to change the
object quickly.  Grip support currently exists for lines, nets, pins,
buses, circles, and boxes.  Arcs do not yet have grips, but will
eventually have them.

Using grips is easy:
\begin{enumerate}
\item Select the object you want to change.  The grips (the little
  boxes) will appear.
  
\item Click and hold the first mouse button inside the box.
        
\item Move the mouse around till you have the object where you want it
  
\item Release the mouse button.
\end{enumerate}

\section{Files used/created by gschem}
There are several files which gschem uses.  Here is a list and a brief
explanation of each:
\begin{itemize}
\item {\tt *.sch} Schematic files.  These files contain components,
  nets, text, and sometimes primitive objects (like lines, circles,
  box etc...)  Schematics do not contain pins.  Schematic filenames
  should follow this convention:  {\tt name\_\#.sch} where:
  \begin{itemize}
  \item {\tt name} is a text string which describes what this schematic contains.
  
  \item {\tt \_\#} is an underscore and a number (like {\tt \_1}, {\tt
      \_2}, {\tt \_7}, {\tt \_13}, etc...)  This number is used to
    sequence schematic pages in a multiple page schematic.
  
  \item {\tt .sch} is the schematic extension/suffix.  It is important
    the schematic pages have this extension.
  \end{itemize}
  
  Schematic files are pure ASCII and will always be pure ASCII.  gEDA
  does not support any binary file formats.  The file format for
  schematics is described in the gEDA file formats document.

  
\item {\tt *.sym} Symbol files.  The schematic and symbol file formats
  are identical.  gschem (or a text editor) is used to create symbol
  files as well as schematics.  Symbol files contain lines, circles,
  boxes, arcs, pins, text, and attributes.
  
  The naming convention for symbol files is:  {\tt name-\#.sym}  where:
  \begin{itemize}
  \item {\tt name} is a text string which describes what the symbol
    represents.
    
  \item {\tt -\#} is a dash and a number (like {\tt -1}, {\tt -2}
    etc...)  The number is used to allow for a symbols to have the
    same name yet different contents.  There might be multiple
    representations for resistors so these symbols should be called:
    {\tt resistor-1.sym}, {\tt resistor-2.sym}, and {\tt resister-3.sym}.
    
  \item {\tt .sym} is the symbol extension/suffix.  It is important
    the symbols have this extension.
  \end{itemize}
  
  The way of specifying hierarchy is by using the {\tt source= attribute}.
  Please see the master attribute document for info on this mechanism.
  
  The hierarchy mechanism is still in heavy flux, so there might be
  some more changes.
  
  \item {\tt system-gschemrc}\\ {\tt ~/.gEDA/gschemrc}\\ {\tt gschemrc}
  Resource files.  These contain guile (scheme) keywords which control
  the various configurable parameters.  All three files may exist and
  are searched in the above listed order.  {\tt system-gschemrc}
  contains system wide defaults.  {\tt ~/.gEDA/gschemrc} lives in each
  users home directory and contains project wide defaults.  {\tt
    gschemrc} lives in each project directory and contains project
  specific defaults.  See the section below on the resource file for
  more info.
  
\item {\tt gschem.log} Log file.  This file contains informative,
  error, warnings etc... messages when gschem was run.  This file is
  created in the working directory that gschem was started in.  This
  allows the user to preserve log files between independent projects.

\end{itemize}
        TBA: Suggested format for projects

        

\section{Electrical connectivity }
As you draw schematics you need be aware of what is considered
to be electrically connected by the gEDA programs.

Nets which are visually connected to other nets are electrically
connected.  This connection may be endpoint to endpoint or endpoint to
midpoint.  When a single endpoint to endpoint (net or pin endpoint)
connection is drawn then the visual dangling net cue disappears.  When
an endpoint ends in the middle of another net (or multiple endpoints
coming together at a single point) then a circular filled connectivity
cue is drawn.  You cannot connect a net to the middle of a pin.  Nets
can only be connected to the endpoints of pins. You cannot connect to
a net if that net is not orthogonal (horizontal or vertical).  The
visual cues are the primary way of telling if nets/pins are connected.

Bus are similar to nets with the exception that you cannot connect a
net to the endpoint of a bus (only to the middle).  If you do try to
connect a net to the end of a bus you will see a big red X at the
invalid endpoint connection.  Buses are still very new so there are
still many quirks.

You can label nets by using the {\tt label=} attribute.  Do not attach more
than one {\tt label=} to a net. You only need to attach the {\tt label=} attribute
to one net segment.  Different nets (i.e. multiple net segments which
aren't connected together) which have the same attribute {\tt label=}
attached to them are also considered electrically connected.  You will
not get any indication of this connection by gschem, but the netlister
(gnetlist) considers nets with the same {\tt label=} attribute electrically
connected.  The naming convention for buses has not been formalized
yet.

More TBA

\section{Dialog boxes}
TBA

\section{Components/symbols}
A component or symbol represents something.  Usually it represents a
gate, a black box or block, or an entire device.  A symbol is a
collection of primitives which are grouped together. You can use
lines, boxes, circles, arcs, text, attributes and pins as the
primitives for building symbols.  You cannot have nets, buses, or
other symbols inside a component.

Components are searched for by specifying (component-library "...")
inside one of the *rc files.  See below for more info. 

More to TBA

\section{Attributes}
An attribute is nothing more than a text item which is in the form
{\tt name=value} (there cannot be any spaced to the left or right of the
name,value pair).  It can be either unattached or attached.
Attributes are used extensively in the gEDA project to convey
information.  Things like device name, pin numbers, hidden nets, and
unit reference numbers are specified using attributes.  For a list of
attributes be sure to look at the attributes.txt document.

There are two kinds of attributes:
\begin{enumerate}
\item Regular attached attributes.  These are attributes which take on
the standard form and are attached to some object (pin, net,
component, or box etc...).  These attributes are usually yellow in
color.
                
\item Unattached attributes.  These are attributes which take on the
standard form, but are not attached to any object.  These attributes
are also known as floating or toplevel attributes.
\end{enumerate}

Regular attached attributes are attached to an object to associate the
info with a specific object.  For example: a pin number associated
with a pin.  Unattached attributes usually convey some information
which is global in nature.  For example: a {\tt device=} attribute (which
lives inside symbols) and specifies what device the entire symbol
represents.

There is a third type of attribute which is a special case of \#2 but
turns into \#1.  This special type of attribute is known as a prompted
attribute.  If you place an unattached visible attribute inside a
symbol and then instantiate that symbol, then that unattached
attribute gets "promoted" to an attached attribute.  This newly
promoted attribute gets attached to the symbol.  This mechanism of
attribute reattachement (from within a symbol) is known as attribute
promotion.

There are some gotchas about attribute promotion:
\begin{itemize}
\item Promotion *only* happens when the symbol is placed.  That means
that if you place a symbol (sym1) and then change it on disk (by
adding or removing floating attributes), existing sym1's will not
reflect these new floating attributes (ie they won't be promoted) in
any schematic.

\item The {\tt device=} attribute is not promoted. 

\item Invisible attributes are not promoted by default.  If you attach
a floating attribute (like {\tt numslots=\#}) and make it invisible, it will
not be promoted.  Now, in order to make everybody happy, this behavior
is configurable.  If you add:
\begin{verbatim}
                (promote-invisible "enabled")
\end{verbatim}
to the {\tt *gschemrc} files (or editing {\tt system-gschemrc}),
invisible floating attributes will also be promoted (and in memory
removed)

However, if you enable this, then component slotting will break,
because gschem expects certain floating attributes to be inside the
symbol (in memory even though they are invisible).  So you can add:
\begin{verbatim}

                (keep-invisible "enabled")
\end{verbatim}
to the {\tt *gschemrc} files (or editing {\tt system-gschemrc}).  This is enabled
by default, but has no effect unless promote-invisible is enabled.
\end{itemize}

So, to summarize, attribute promotion takes floating attributes inside
symbols and attaches them to the outside of a placed symbol.  Three
*rc keywords control this behavior: attribute-promotion,
promote-invisible, and keep-invisible.

\section{Resource File}
Gschem is highly configurable.  All configuration is handled through
a scheme based rc file.  Gschem looks for three rc files:

\begin{itemize}
\item {\tt system-gschemrc}:  This is usually installed in 
                  {\tt /usr/share/share/gEDA} and is required for gschem to run.

\item {\tt \$HOME/.gEDA/gschemrc}: A per user file.  Users should put
                  defaults in this file they want to apply to all sessions. 

\item local directory {\tt gschemrc}: This file holds the per project
                  defaults.  Things like component-library or source-library
                  keywords go into this file.

\end{itemize}
The order of searching for these three files is as above (first {\tt
  system-gschemrc}, then {\tt \$HOME/.gEDA/gschemrc}, and finally the
local {\tt gschemrc}).

A few rules about changing the files:
\begin{itemize}
\item Don't break any syntax rules.  Doing so will cause the scheme
  interpreter (guile) to stop interpreting.
  
\item Keywords/defaults always override what came before, with the
  exception of cumulative keywords (like component-library).
\end{itemize}

For more info as to what can be configured, please look in
{\tt system-gschemrc}.

More TBA.
        
\end{document}
