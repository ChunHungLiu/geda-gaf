% gEDA - GPL Electronic Design Automation
% attributes.tex - Master Attribute Document
% Copyright (C) 2002 Ales V. Hvezda
%
% This program is free software; you can redistribute it and/or modify
% it under the terms of the GNU General Public License as published by
% the Free Software Foundation; either version 2 of the License, or
% (at your option) any later version.
%
% This program is distributed in the hope that it will be useful,
% but WITHOUT ANY WARRANTY; without even the implied warranty of
% MERCHANTABILITY or FITNESS FOR A PARTICULAR PURPOSE.  See the
% GNU General Public License for more details.
%
% You should have received a copy of the GNU General Public License
% along with this program; if not, write to the Free Software
% Foundation, Inc., 675 Mass Ave, Cambridge, MA 02139, USA.

\documentclass{article}

\setlength{\parindent}{0pt}
\setlength{\parskip}{1ex plus 0.5ex minus 0.2ex}

\title{gEDA/gaf Master Attribute Document}
\author{Ales V. Hvezda, ahvezda@geda.seul.org\\
	\\
	This document is released under GFDL\\ 
	(http://www.gnu.org/copyleft/fdl.html)}
\date{February 23rd, 2003}

\begin{document}

\maketitle
\newpage

\tableofcontents
\newpage


\section{Overview}

This document describes all the attributes used in in gEDA/gaf (GPL'd
Electronic Design Automation / Gschem And Friends).  This document
is broken down into several section: this overview, symbol only
attributes, schematic only attributes, attributes which can appear
in both symbols and schematics, and attributes which are obsolete or
deprecated.

In this document, attribute names are in {\bf bold} and examples
are in the \texttt{typewriter} font.

\section{What are Attributes?}

Attributes in the gEDA/gaf system are nothing more than text items
which take on the form: {\bf name}=value.  Name can be anything just as long
as it doesn't contain a equals sign.  Value can also be anything just
as long as it is something (vs nothing).  {\bf name}= (without a value part)
is not a valid attribute.  Also, there cannot be any spaces immediately
before or after the equals sign.

Attributes can be attached to some part of the symbol.  If the attribute
conveys information specific to an object, then the attribute should
be attached directly to the object, otherwise the attribute should be
free standing or floating.  Free standing attributes just exist in the
symbol file as text items which take on the form {\bf name}=value.

\newpage

\section{Symbol only Attributes}


\subsection{\bf device\label{device}}
{\bf device}= is the device name of the symbol and is required by gnetlist.  

{\bf device}= should be placed somewhere in the symbol and made invisible.
This is a free standing or floating attribute.  If the object is a graphic
then {\bf device}= should be set to none ({\bf device}=none) and attach
a {\bf graphical}= (\ref{graphical}) attribute.  Do not confuse this
attribute with just having a text label which the device name.  Do not
put spaces into the device name; there are some programs which dislike
spaces in the device specifier.  Generally the device name is in all caps.

Examples: \texttt{device=7400 device=CONNECTOR\_10 device=NPN\_TRANSISTOR}


\subsection{\bf graphical\label{graphical}}
Symbols which have no electrical or circuit significance need a 
{\bf graphical}=1 attribute.  Symbols like titleboxes are purely
graphical symbols.  Any symbol which has {\bf graphical}=1 is ignored
by gnetlist.

{\bf graphical}=1 should exist somewhere in the symbol and made invisible.
This is a free standing or floating attribute.  Don't forget to set 
{\bf device}=none (\ref{device}).

Example: \texttt{graphical=1}


\subsection{\bf description\label{description}}
The {\bf description} attribute provides a simple one line description of what 
the symbol is supposed to represent.  

Example: \texttt{description=4 NAND gates with 2 inputs}


\subsection{\bf author\label{author}}
The {\bf author} attribute identifies the name of the author of this
symbol.  This attribute is optional, but it is nice to know who create
which symbols.  This symbol is fairly free form and it can also include
people's names who modified the symbol.

Example: \texttt{author=Ales Hvezda}


\subsection{\bf email\label{email}}
The {\bf email} attribute identifies the email of the author of this
symbol.  This attribute is also optional, but keeping track of who created
the symbol and a contact address is very useful if there are questions
about the symbol, which only the original author might be able to answer.
It can also include the email address of who else modified the symbol.
It is probably also a good idea to obfuscate the address so it is not
harvested for spam purposes.

Example: \texttt{email=ahvezdaATseul.org}


\subsection{\bf comment\label{comment}}
The {\bf comment} attribute can contain anything.  This attribute can
convey any additional information which might not fit into any other
attribute.  There can be multiple instances of this attribute.

Example: \texttt{comment=This is a comment inside a symbol}


\subsection{\bf pinseq\label{pinseq}}
This attribute is used to give each pin an unique number or sequence.
All pins must have a {\bf pinseq}=\# attribute attached to the pin object.
This attribute should be hidden.  This attribute is used extensively by
gschem and gnetlist.

gnetlist will output pins in the order of increasing pin sequence.
The sequence numbers start at 1 and should increase without skipping
any numbers.  This attribute is not the pin number (i.e. device pin
numbers, like GND is 7 on TTL).  For pin numbers see the {\bf pinnumber} (\ref{pinnumber}) attribute.  

Examples: \texttt{pinseq=1 pinseq=2 pinseq=3}

This attribute replaces the obsolete {\bf pin\#}=\# attribute.


\subsection{\bf pinnumber\label{pinnumber}}
This attribute is the pin number (i.e. like GND is 7 on 74 TTL).  
All pins must have a {\bf pinnumber}=\# attribute attached to the pin object.

You can have numbers or letters for the value.  This attribute should
be visible with the value only visible.  You also need a {\bf pinseq}
(\ref{pinseq}) attribute.

Examples: \texttt{pinnumber=1 pinnumber=13 pinnumber=A0}   

This attribute replaces the obsolete {\bf pin\#}=\# attribute.


\subsection{\bf pintype\label{pintype}}
Each pin must have a {\bf pintype}=value attribute attached to it and 
should be make hidden. Table \ref{pintype values} shows valid values for
this attribute.  

\vspace{.125in}

\begin{table}[h]
\begin{center}
\begin{tabular}{|l|l|} \hline
in & Input \\ \hline
out & Output \\ \hline
io & Input/Output \\ \hline
oc & Open collector \\ \hline
oe & Open emitter \\ \hline
pas & Passive \\ \hline
tp & Totem pole \\ \hline
tri & Tristate (high impedance) \\ \hline
clk & Clock \\ \hline
pwr & Power/Ground \\ \hline
\end{tabular}
\end{center}
\caption{{\bf pintype} values} \label{pintype values}
\end{table}

\vspace{.125in}

This attribute is not used extensively in the symbol library, but it
will be used for DRC and netlisting.

Examples: \texttt{pintype=clk pintype=in pintype=pas}   


\subsection{\bf pinlabel\label{pinlabel}}
This attribute labels a pin object.  This attribute is primarily used 
by gnetlist to support hierarchical designs.

This attribute must be attached to the pin and be left visible.  Please
make this attribute green (instead of the default attribute yellow).

Examples: \texttt{pinlabel=A0 pinlabel=DATA1 pinlabel=CLK}


\subsection{\bf numslots\label{numslots}}
If a component has multiple slots in a physical package (such as a 7400
(NAND) which has 4 NANDs per package) then you need a {\bf numslots}=\#
attribute.  The \# is the number of slots that are in a physical device.
{\bf numslots}=\# should exist somewhere in the symbol and be made
invisible.  This is a free standing or floating attribute.  If the symbol
does not need slotting, then put {\bf numslots}=0 into the symbol file.

Example: \texttt{numslots=4}


\subsection{\bf slotdef\label{slotdef}}
If a component has multiple slots in a physical package then you must 
attach a {\bf slotdef}=slotnumber:\#,\#,\#... for every device inside
the physical package.

The slotnumber corresponds to the slot number.  The colon after the
slot number is required.  For example, if a device has 4 slots then there
would be {\bf slotdef}=1:..., {\bf slotdef}=2:..., {\bf slotdef}=3:...,
and {\bf slotdef}=4:...  attributes somewhere in the symbol and be
made invisible.  This is a free standing or floating attribute.

The \#'s have a one-to-one correspondence to the {\bf pinseq} attributes
and specify which {\bf pinnumber}=\# is used during display (gschem)
or netlisting (gnetlist).

It is recommended that all symbols which have slots have a {\bf slot}=1
(\ref{slot}) attribute attached in the same fashion as the {\bf device}=
(\ref{device}) attribute.

See 7400-1.sym as a concrete example.

Examples: \texttt{slotdef=1:1,2,3 slotdef=2:4,5,6 slotdef=3:7,8,9}

This attribute replaces the obsolete {\bf slot}\#=\# attribute.


\subsection{\bf footprint\label{footprint}}
{\bf footprint}=package\_name should exist somewhere in the symbol and
be made invisible.  This attribute is used by gnetlist and primarily
for the PCB package.

Attach this attribute just like the {\bf device}= (\ref{device})
attribute.  This is a free standing or floating attribute.

package\_name is the pcb footprint or package type like DIP14 or DIP40.
Although this attribute in principle is pcb package dependent, 
gEDA/gaf conventions exist to make this attribute as portable as
possible, allowing for easy collaboration and sharing between users.
See the {\it Footprint naming conventions} in the
{\it gEDA/gaf Symbol Creation Document}.

\subsection{\bf documentation\label{documentation}}
{\bf documentation}=documentation\_locator may exist somewhere in the 
symbol and be made invisible.  This attribute is used by gschemdoc to 
find relevant documentation for the symbol, or rather, the device or 
component associated with the symbol.

Attach this attribute just like the {\bf device}= (\ref{device})
attribute.  This is a free standing or floating attribute.

documentation\_locator is either the base filename of the documentation,
or it is the complete Internet URL (Uniform Resource Locator). If it
is the filename, an attempt will be made to search for it in the local
gEDA share directory named \texttt{documentation}.

Filename example: \texttt{documentation=sn74ls00.pdf}

URL example: \texttt{documentation=http://www-s.ti.com/sc/ds/sn74ls00.pdf}


\section{Schematic only Attributes}

\subsection{{\bf netname}\label{netname}}
This attribute should be attached to a net object to give it a name.
Multiple net names for connected net segments is discouraged.  All nets
which have the same value are considered electrically connected.  This
attribute is not valid inside symbols (as you cannot have nets inside
of symbols).

Examples: \texttt{netname=DATA0\_H netname=CLK\_L}


\subsection{{\bf source}\label{source}}
The {\bf source}= attribute is used to specify that a symbol has underlying
schematics.  This attribute is attached directly to a component.

This attribute should only be attached to instantiated components
in schematics.  Attach the attribute to a component and specify the
filename (not the path) of the underlying schematic (like block.sch)
for the value.  The specified schematic must be in a source-library path.
This attribute can be attached multiple times with different values
which basically means that there are multiple underlying schematics.

Examples: \texttt{source=underlying.sch source=memory.sch}


\section{Symbol and Schematic Attributes}

\subsection{{\bf refdes}\label{refdes}}
This attribute is used to specify the reference designator to a particular
instantiated component.  It must be on ALL components which have some
sort of electrical significance.  This attribute can also be on the
inside of a symbol (it will be promoted, i.e. attached to the outside of
the symbol, if it is visible) to provide a default refdes value (such as
U?).

Examples: \texttt{refdes=U1 refdes=R10 refdes=CONN1}

\subsection{{\bf slot}\label{slot}}
This attribute is used to specify a slot for a slotted component.
It should be attached to an instantiated component.  This attribute can
also be on the inside of a symbol (it will be promoted, i.e. attached
to the outside of the symbol, if it is visible) to provide a default
slot.

\subsection{{\bf net}\label{net}}
The {\bf net}= attribute is used to create power/ground and arbitrary nets.
Please see the netattrib Mini-HOWTO for more info.  When this attribute
is inside a symbol, it is used to create nets.  When this attribute is
attached to an instantiated component (in a schematic), then the
{\bf net}= can also be used to create new nets and can used to override
existing nets.

\subsection{{\bf value}\label{value}}
Used mainly in the spice backend netlister to specify the value of the
various elements.  No translation is done on this, and it is placed as
is into the netlist.

Examples: \texttt{value=1K value=10V}


\section{Obsolete Attributes}

\subsection{{\bf uref}\label{uref}}
The {\bf uref}= attribute is obsolete and cannot not be used.  It was used
to provide the same information as {\bf refdes} (\ref{refdes}).


\subsection{{\bf name}\label{name}}
The {\bf name}= attribute should not be attached or appear in any symbol.
It is considered ambiguous.  {\bf name}= was never used by gEDA/gaf.


\subsection{{\bf label}\label{label}}
The {\bf label}= attribute is obsolete and cannot be used.  It was 
used to give nets names/labels and to label pins.  The replacement
attributes for this are {\bf netname} (\ref{netname}) and 
{\bf pinlabel} (\ref{pinlabel}) respectively.


\subsection{{\bf pin\#}\label{pinPOUND}}
The {\bf pin\#}=\# attribute is obsolete and cannot be used.  It was
used to provide sequence and number information to pins.  The replacement
attributes for this are {\bf pinseq} (\ref{pinseq}) and {\bf pinnumber}
(\ref{pinnumber}).


\subsection{{\bf slot\#}\label{slotPOUND}}
The {\bf slot\#}=\# attribute is obsolete and cannot be used.  It was
used to provide slotting information to components.  The replacement
attribute for this is {\bf slotdef} (\ref{slotdef}).


\subsection{{\bf type}\label{type}}
The {\bf type=} attribute is obsolete and cannot be used.  It was
used to provide type information on pins. The replacement
attribute for this is {\bf pintype} (\ref{pintype}).


\newpage
\section{Document Revision History}

\begin{table}[h]
\begin{tabular}{|l|l|} \hline
July 14th, 2002 & Created attributes.tex from attributes.txt. \\ \hline
July 14th, 2002 & Updated doc to be in sync with post-20020527. \\ \hline
August 25th, 2002 & Added obsolete type= attribute. \\ \hline
September 14, 2002 & Added description= attribute. Minor fixes\\ \hline
October 7, 2002 & Added doc= attribute; Egil Kvaleberg.\\ \hline
February 11, 2002 & Added reference to footprint conventions.\\ \hline
February 23, 2002 & Added author=, email=, and comment= attributes.\\ \hline

\end{tabular}
\end{table}

\end{document}

