% $Id$
%
% gEDA - GPL Electronic Design Automation
% geda_pads_annotate.tex - Documentation for forward/backward
% annotation between gEDA and Pads PowerPCB.
%
% Copyright (C) 2003 Dan McMahill
%
% This program is free software; you can redistribute it and/or modify
% it under the terms of the GNU General Public License as published by
% the Free Software Foundation; either version 2 of the License, or
% (at your option) any later version.
%
% This program is distributed in the hope that it will be useful,
% but WITHOUT ANY WARRANTY; without even the implied warranty of
% MERCHANTABILITY or FITNESS FOR A PARTICULAR PURPOSE.  See the
% GNU General Public License for more details.
%
% You should have received a copy of the GNU General Public License
% along with this program; if not, write to the Free Software
% Foundation, Inc., 675 Mass Ave, Cambridge, MA 02139, USA.

\documentclass{article}
%\usepackage{epsfig}
\usepackage{graphicx}
% This line will enable hyperlinks in the PDF output
% file.
\usepackage[ps2pdf,breaklinks=true,colorlinks=true]{hyperref}

\setlength{\parindent}{0pt}
\setlength{\parskip}{1ex plus 0.5ex minus 0.2ex}

\title{Forward/Backward Annotation Between gEDA/gaf and Pads PowerPCB}
\author{Dan McMahill\\
        \\
        This document is released under GFDL\\ 
	(\url{http://www.gnu.org/copyleft/fdl.html})}
\date{March 6th, 2003}

\begin{document}

\maketitle
\newpage

\tableofcontents
\newpage

\section{Forward Annotation of gEDA Schematic Changes to Pads PowerPCB Layout}
\subsection{Overview}
Forward annotation is the process of updating a layout to reflect
changes made in the schematic.  This process is used when, for
example, a new component is added to a schematic and needs to be
included in the layout.  This section describes how to forward
annotate changes in a gEDA schematic to a Pads PowerPCB layout.

Pads implements forward annotation through the use of an ECO
(Engineering Change Order) file.  The ECO file describes the
differences between a current design and the desired design.  Pads
generates the ECO file by performing a netlist comparison between a
new netlist file and the netlist contained in the current layout.

\subsection{Detailed Forward Annotation Procedure}
This procedure assumes you have a board layout open in Pads and that
you have made your schematic changes in {\tt gschem}.  For the
purposes of illustration, assume your schematic is split into two
pages in the files {\tt pg1.sch} and {\tt pg2.sch}.

\begin{enumerate}
\item Create an updated Pads netlist by running ``{\tt gnetlist -g pads
    -o mynet.asc pg1.sch pg2.sch}''.  This will create the netlist
  file ``{\tt mynet.asc}''.

\item Make a backup copy of your Pads layout in case things fail in a
  destructive way.

\item From within Pads, choose the ``Tools$\rightarrow$Compare
  Netlist'' menu item and choose the following options in the form.\\
  \\
  \begin{tabular}{|r l|}
    \hline 
    original design to compare:& use current PCB design \\
    new design with changes:& mynet.asc \\
     $\surd$&generate differences report \\
     $\surd$&generate eco file\\
     &\\
     comparison options &\\
     $\surd$&compare only ECO registered parts\\
     &\\
     attribute comparison level &\\
     $\surd$&ignore all attributes\\
    \hline 
  \end{tabular}
  \\
  Click the OK button to create the ECO file.

\item Examine the ECO file to make sure it looks ok (the ECO file is a
  text file which can be viewed with any text editor).

\item From within Pads, choose the ``File$\rightarrow$Import...''
  menu item.  Locate and choose the ECO file created previously.

\end{enumerate}

\section{Back Annotation of Pads PowerPCB Layout Changes to gEDA Schematic}
Backannotation is the process of updating schematics to reflect
changes made in the layout.  This process is used, for example, when
the reference designators have been renumbered on the layout, when
pins have been swapped (e.g. on an AND gate), or slots have been
swapped (e.g. on a multi-gate package).  This section describes how to
backannotate changes in a Pads PowerPCB layout to a gEDA schematic.
The Pads PowerPCB tool supports three types of schematic
backannotation:

\begin{enumerate}
\item Reference designator changes.  This is often times used at the
end of a layout to give components which are geographically close a
set of reference designators which are numerically close.

\item Slot swapping.  This is commonly found in digital designs where
there may be multiple identical gates in a single package.  For
example, you may wish to swap which slot is used in a hex inverter.

\item Pin swapping.  During layout, the designer may wish to swap
equivalent pins on a chip.  For example, the two inputs on a NAND
gate.
\end{enumerate}

Currently only reference designator changes are automatically
processed by the Pads to gschem backannotation tool.  The slot and pin
swapping changes are provided in a report which the schematic designer
must use to manually correct the schematic.

\subsection{Detailed Backannotation Procedure}
This procedure assumes you have a board layout open in Pads.  For the
purposes of illustration, assume your schematic is split into two
pages in the files {\tt pg1.sch} and {\tt pg2.sch}.

\begin{enumerate}
\item Create an up to date Pads netlist by running ``{\tt gnetlist -g pads
    -o mynet.asc pg1.sch pg2.sch}''.  This will create the netlist
  file ``{\tt mynet.asc}''.

\item From within Pads, choose the ``Tools$\rightarrow$Compare
  Netlist'' menu item and choose the following options in the form.\\
  \\
  \begin{tabular}{|r l|}
    \hline 
    original design to compare:& mynet.asc \\
    new design with changes:& use current PCB design \\
     $\surd$&generate differences report \\
     $\surd$&generate eco file\\
     &\\
     comparison options &\\
     $\surd$&compare only ECO registered parts\\
     &\\
     attribute comparison level &\\
     $\surd$&ignore all attributes\\
    \hline 
  \end{tabular}
  \\
  Click the OK button to create the ECO file.

\item Examine the ECO file to make sure it looks ok (the ECO file is a
  text file which can be viewed with any text editor).

\item Make a backup copy of your gEDA schematic files in case things fail
  in a destructive way.

\item Run ``{\tt pads\_backannotate file.eco pg1.sch pg2.sch | tee
    backanno.log}'' where 
  {\tt file.eco} is the name of the ECO file created previously and
  {\tt pg1.sch} and {\tt pg2.sch} are all of your schematic pages.
  This will apply the reference designator change portion of the ECO
  file and also generate a list of pin and slot swapping which must be
  performed by hand.  The file {\tt backanno.log} will contain a log
  of the session that can be refered to when performing the pin and
  slot swapping.
\end{enumerate}

\end{document}

